\chapter{Aplicaciones lineales}

\begin{defi}
	Una aplicación $f: U \longrightarrow V$ entre $\mathbb{K}$-espacios vectoriales es \textbf{una aplicación lineal} si para todo $u,w \in U$ y todo $\alpha \in \mathbb{K}$ se cumplen las siguientes propiedades:
	\begin{enumerate}
		\item $f(u+v) = f(u) + f(v)$.
		\item $f(\alpha u) = \alpha f(u)$.
	\end{enumerate}
	O equivalentemente, si $\forall u,v \in U,\forall \alpha,\beta \in \mathbb{K}$ se cumple:
	\[
	f(\alpha u + \beta v) = \alpha f(u) + \beta f(v)
	\]
\end{defi}

Denotamos $\mathcal{L}(U,V)$ como el conjunto de aplicaciones lineales:
\[
\mathcal{L}(U,V) \equiv \lbrace f: U \longleftarrow V: f \ es \ una \ aplicacion \ lineal \rbrace
\]

Toda aplicación lineal queda completamente determinada conociendo las imágenes de los vectores de una base.

\begin{prop}
	Sea $f: U \longrightarrow V:$ una aplicación lineal. Son ciertas las afirmaciones:
	\begin{enumerate}
		\item $f(0_U)=0_V$.
		\item $f(\alpha_1 u_1 + \cdots + \alpha_n u_n) = \alpha_1 f(u_1) + \cdots + \alpha_n f(u_n). \forall \alpha_1,\ldots,\alpha_n \in \mathbb{K},\forall u_1,\ldots,u_n \in U$
	\end{enumerate}
\end{prop}

\begin{prop}
	Sean $U$ y $V$ $\mathbb{K}$-espacios vectoriales, $\mathcal{B}=\lbrace u_1,\ldots, u_n \rbrace$ una base de $U$ y $v_1,\ldots,v_n$ vectores de $V$. Existe una única aplicación lineal $f: U \longleftarrow V$ tal que $f(u_i)=v_i$, para $i=1,\ldots,n$.
\end{prop}

\begin{proof}
	Definimos la aplicación $f$ tal que $f(u_i)=v_i$, para $i=1,\ldots,n$ y que para un vector cualquiera $x=(x_1,\ldots,x_n)_\mathbb{B}$ de $U$ cumple que:
	\[
	f(x)=x_1 f(u_1) + \cdots + x_n f(x_n)
	\] 
	Comprobamos que $f$ es aplicación lineal. Sean $x=()$
\end{proof}