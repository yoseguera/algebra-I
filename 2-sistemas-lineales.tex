\chapter{Sistemas lineales}

Dos sistemas lineales son \textbf{equivalentes} si coinciden en su solución general.

Para $\alpha \in \mathbb{K}, \alpha \neq 0$ son equivalentes los sistemas lineales:
\begin{align*}
\mathcal{A} \equiv \{ a_{11}x_1 + \cdots + a_{1n}x_n = b_1 \\
\mathcal{A'} \equiv \{ \alpha a_{11}x_1 + \cdots + \alpha  a_{1n}x_n = \alpha b_i
\end{align*}

Para $\alpha \in \mathbb{K}, \alpha \neq 0$ son equivalentes los sistemas lineales:

\begin{align*}
\mathcal{A} \equiv & \begin{cases}
a_{11}x_1 + \cdots + a_{1n}x_n = b_1 \\
a_{21}x_1 + \cdots + a_{2n}x_n = b_2 
\end{cases} \\
\mathcal{A'} \equiv & \begin{cases}
a_{11}x_1 + \cdots + a_{1n}x_n = b_1 \\
(a_{21} + \alpha a_{11})x_1 + \cdots + (a_{2n} + \alpha a_{1n})x_n = b_2 
\end{cases} \\
\end{align*}

Dos sistemas lineales con matrices ampliadas equivalentes por filas son sistemas equivalentes.

\section{Discusión y resolución de sistemas lineales}

Un sistema lineal $\mathcal{A}$ se caracteriza por el número de soluciones:
\begin{itemize}
\item $\mathcal{A}$ es \textbf{incompatible} si no tiene solución.
\item $\mathcal{A}$ es \textbf{compatible determinado} si tiene una única solución.
\item $\mathcal{A}$ es \textbf{compatible indeterminado} si tiene infinitas soluciones.
\end{itemize}

\textbf{Resolver} $\mathcal{A}$ es encontrar la solución general, y \textbf{discutir} $\mathcal{A}$ es determinar si es incompatible, compatible determinado o indeterminado.

En un \textbf{sistema escalonado}:
\begin{itemize}
\item Si aparece $b=0$ con $b \neq 0$ entonces es incompatible.
\item Si aparece $0=0$, una fila nula, se puede eliminar.
\end{itemize}

\subsection{Discusión de sistemas lineales}

Sea $\mathcal{A}$ un sistema lineal escalonado con $n$ incógnitas y sea $(A|B)$ su matriz ampliada.
\begin{enumerate}
\item Si $(A|B)$ tiene un pivote en su última columna entonces $\mathcal{A}$ es incompatible.
\item Si $(A|B)$ no tiene pivote en la última columna:
\begin{enumerate}
\item $(A|B)$ tiene $n$ entonces es compatible determinado.
\item $(A|B)$ tiene menos $n$ entonces es compatible indeterminado.
\end{enumerate}
\end{enumerate}

\subsection{Teorema de Rouche-Fröbenius}

Sea $\mathcal{A}$ un sistema lineal con $n$ incógnitas y sea $(A|B)$ su matriz ampliada:
\begin{enumerate}
\item $\mathcal{A}$ es incompatible si y sólo si $rg(A) < rg(A|B)$.
\item $\mathcal{A}$ es compatible determinado si y sólo si $rg(A) = rg(A|B) = n$.
\item $\mathcal{A}$ es compatible indeterminado si y sólo si $rg(A) = rg(A|B) < n$.
\end{enumerate}

Un sistema lineal $AX=B$ con $A$ de orden $n$ tiene solución única si y sólo si $A$ es invertible.

\subsection{Regla de Cramer}

La única solución de un sistema regular de orden $n$ $AX=B$ viene dada por:
\[
x_1=\frac{\vartriangle_1}{det(A)}, x_2=\frac{\vartriangle_2}{det(A)}, \ldots, x_n=\frac{\vartriangle_n}{det(A)}
\]
donde $\vartriangle_i$ es el determinante de la matriz que se obtiene cambiando la columna $i$ de $A$ por $B$.